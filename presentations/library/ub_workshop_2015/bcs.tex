\section{Essential BCs}
% Auto-generate the TOC slide(s)
\begin{frame}
  \tableofcontents[currentsection]
  %\tableofcontents
\end{frame}



\subsection*{Essential Boundary Data}
\begin{frame}[t]
  %\vspace{-.2in}
  \begin{block}{
      %Essential Boundary Data
    }
  \begin{itemize}
  \item {Dirichlet boundary conditions can be enforced after 
    the global stiffness matrix $\bv{K}$ has been assembled}
  \item This usually involves
    \begin{enumerate}
    \item<1-> placing a ``1'' on the main diagonal of the
      global stiffness matrix
    \item<2-> zeroing out the row entries
    \item<3-> placing the Dirichlet
      value in the rhs vector
    \item<4-> subtracting off the column entries from the rhs
    \end{enumerate}
  \end{itemize}
  \end{block}
  \visible<5->{
    \vspace{-0.1in}
  \begin{equation}
    \nonumber
      \begin{bmatrix}
	k_{11} & k_{12} & k_{13} & .  \\
	k_{21} & k_{22} & k_{23} & .  \\
	k_{31} & k_{32} & k_{33} & .  \\
	  .    &   .    &    .   & .  
      \end{bmatrix},
      \begin{bmatrix}
	f_{1}  \\
	f_{2}  \\
	f_{3}  \\
	  .     
      \end{bmatrix} \rightarrow
      \begin{bmatrix}
	1      & 0      & 0      & 0  \\
	0      & k_{22} & k_{23} & .  \\
	0      & k_{32} & k_{33} & .  \\
	  0    &   .    &    .   & .  
      \end{bmatrix},
      \begin{bmatrix}
	g_{1}  \\
	f_{2} - k_{21}g_1  \\
	f_{3} - k_{31}g_1  \\
	  .     
      \end{bmatrix}      
  \end{equation}}

\end{frame}



\begin{frame}[c]
%\begin{block}{}
  \begin{itemize}[<+->]
    \item {Cons of this approach :
      \begin{itemize}[<+->]
      \item {Works for an interpolary finite element basis
	but not in general.}
	
      \item {May be inefficient to change individual entries once the global matrix is assembled.}
      \end{itemize}
      }
    \item {Need to enforce boundary conditions for
      a generic finite element basis \emph{at the element stiffness matrix level}.}

    %\item Solution: ``Penalty'' Boundary Conditions
  \end{itemize}
%  \end{block}
\end{frame}


\subsection*{Penalty Formulation}
\begin{frame}[c]
%\begin{block}{}
  \begin{itemize}[<+->]
  \item {One solution is the ``penalty'' boundary formulation}
    %
  \item {A term is added to the standard weighted residual statement
    \begin{equation}
      \nonumber
      (F( u ), v)
      + \underbrace{\frac{1}{\epsilon} \int_{\partial \Omega_D} (u-u_D)v \; dx}_{\text{penalty term}} =
      0 \hspace{.3in} \forall v \in \mathcal{V}
    \end{equation}
  }
    %
  \item {Here $\epsilon \ll 1$ is chosen so that, in floating point arithmetic,
    $\frac{1}{\epsilon} + 1 = \frac{1}{\epsilon}$.}
    %
  \item {This weakly enforces $u=u_D$ on the Dirichlet boundary, and works for
    general finite element bases.}

%%   \item It requires a few additional calculations (edge/face integrals) but is more
%%     efficient than modifying row entries after assembly
  \end{itemize}
%  \end{block}
\end{frame}



\begin{frame}[fragile]
  \begin{block}{}
  \texttt{\libMesh{}} provides:
  \begin{itemize}
  \item {A quadrature rule with \texttt{Nqf} points and \texttt{JxW\_f[]}}
  \item {A finite element coincident with the boundary face that has % \texttt{Nf}
    shape function values \texttt{phi\_f[][]}}
  \end{itemize}
  \end{block}
\small
  \begin{semiverbatim}
for (qf=0; qf<Nqf; ++qf) \{
  for (i=0; i<Nf; ++i) \{
    Fe(i) += JxW_f[qf]*
      \alert<2>{penalty}*\alert<3>{uD(xyz[q])}*phi_f[i][qf];
	
    for (j=0; j<Nf; ++j)
      Ke(i,j) += JxW_f[qf]*
        \alert<2>{penalty}*phi_f[j][qf]*phi_f[i][qf];
  \}
\}
    \end{semiverbatim}

\end{frame}
